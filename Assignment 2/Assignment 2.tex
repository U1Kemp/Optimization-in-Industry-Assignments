\documentclass[11pt]{article}
\usepackage{amsmath,amssymb,amsthm,amsfonts}
\usepackage{geometry}
\usepackage{enumitem}
\usepackage{setspace}
\geometry{margin=1in}
\onehalfspacing

\begin{document}

\title{Optimization in Industry Assignment 2
\\Mathematical Model for Optimal ATM Refill Problem}
\author{Utpalraj Kemprai\\
MDS202352}
\date{}
\maketitle

\section*{Sets}
\begin{itemize}[leftmargin=1.5cm]
    \item[{1.}] \(\mathcal{I}\): Set of ATMs, indexed by \(i\) (i.e., \(i=1,\dots,|\mathcal{I}|\)).
    \item[{2.}] \(\mathcal{V}\): Set of vehicles, indexed by \(v\) (i.e., \(v=1,\dots,|\mathcal{V}|\)).
    \item[{3.}] \(\mathcal{D}\): Set of days in the planning horizon, indexed by \(d\) (e.g., \(d=1,\dots,7\)).
    \item[{4.}] \(\mathcal{K}\): Set of denominations, indexed by \(k\).
\end{itemize}

\section*{Parameters}
\begin{enumerate}[leftmargin=1.5cm]
    \item \(\displaystyle \text{Demand}_{i,d}\): Cash demand at ATM \(i\) on day \(d\).
    \item \(\displaystyle I_i^0\): Initial cash inventory at ATM \(i\) (at start of day 1).
    \item \(\displaystyle C_i\): Maximum cash capacity of ATM \(i\).
    \item \(\displaystyle L_i\): Minimum cash level required at ATM \(i\).
    \item \(\displaystyle Q_{\min}\): Minimum cash deposit per refill (e.g., 50K).
    \item \(\displaystyle Q_{\text{unit}}\): Cash deposit unit (e.g., 10K), so that any nonzero refill is a multiple of \(Q_{\text{unit}}\).
    % \item \(\displaystyle Q_{\min,k}\): Minimum cash deposit per refill (e.g., 50K) for denomination \(k\).
    \item \(\displaystyle Q_{\text{unit},k}\): Cash deposit unit (e.g., 10K) for denomination \(k\), so that any nonzero refill is a multiple of \(Q_{\text{unit},k}\).
    \item \(\displaystyle \text{cap}_v\): Maximum cash carrying capacity of vehicle \(v\) (security limited).
    \item \(\displaystyle \text{cost}_v\): Cost associated with using vehicle \(v\) for a day.
    \item \(\displaystyle \text{maxVisit}_{v,d}\): Maximum number of ATMs vehicle \(v\) can service on day \(d\) (e.g., 20 on weekdays, 30 on weekends).
    \item \(\displaystyle S\): Security gap in days. If vehicle \(v\) visits ATM \(i\) on day \(d\), it cannot visit the same ATM again in the next \(S-1\) days.
    \item \(\displaystyle \Delta_x\): Maximum allowed difference in the number of ATMs visited by any two vehicles in the planning horizon.
    \item \(\displaystyle \Delta_y\): Maximum allowed difference in the cash deposited by any two vehicles in the planning horizon.
    
    \item \(\displaystyle r_{i,d,k}\): Denomination requirement for ATM \(i\) on day \(d\) for denomination \(k\) (soft constraint).
    \item \(\displaystyle w_1,\, w_2,\, w_3, w_4, w_5\): Weights for the objective function corresponding to vehicles used, ATM inventory holding cost, and number of visits, respectively.
    \item \(\displaystyle M \left(= 1 + \max_{v \in \mathcal{V}} \text{cap}_v\right)\): A sufficiently large constant (Big-M constant).
\end{enumerate}

\section*{Decision Variables}
\begin{enumerate}[leftmargin=1.25cm]
    \item \(\displaystyle x_{i,v,d} \in \{0,1\}\): Equals 1 if vehicle \(v\) refills ATM \(i\) on day \(d\); 0 otherwise.
    % \item \(\displaystyle x_{i,v,d,k} \in \{0,1\}\): Equals 1 if refills of denomination \(k\) is made by vehicle \(v\) at ATM \(i\) on day \(d\); 0 otherwise.
    \item \(\displaystyle q_{i,v,d} \in \mathbb{Z}_{\geq 0}\): Number of cash deposit units \(Q_{\text{unit}}\), delivered to ATM \(i\) by vehicle \(v\) on day \(d\).
    \item \(\displaystyle y_{i,v,d} \ge 0\): Amount of cash delivered to ATM \(i\) by vehicle \(v\) on day \(d\).  
    \[
    y_{i,v,d} = Q_{\text{unit}} \cdot q_{i,v,d}, \quad q_{i,v,d} \in \mathbb{Z}_{\geq 0}
    \]
    with the additional condition that if \(q_{i,v,d} > 0\) then \(q_{i,v,d} \ge \frac{Q_{\min}}{Q_{\text{unit}}}\).
    \item \(\displaystyle z_{v,d} \in \{0,1\}\): Equals 1 if vehicle \(v\) is used on day \(d\); 0 otherwise.
    \item \(\displaystyle I_{i,d}\): Cash inventory at ATM \(i\) at the end of day \(d\).
    \item \(\displaystyle q_{i,v,d,k} \in \mathbb{Z}_{\geq 0}\): Number of cash deposit units \(Q_{\text{unit},k}\) of currency \(k\), delivered to ATM \(i\) by vehicle \(v\) on day \(d\).
    \item \(\displaystyle y_{i,v,d,k} \ge 0\): Cash delivered in denomination \(k\) for ATM \(i\) by vehicle \(v\) on day \(d\).
    \[
    y_{i,v,d,k} = Q_{\text{unit},k} \cdot q_{i,v,d,k}, \quad q_{i,v,d,k} \in \mathbb{Z}_{\geq 0}
    \]
    % with the additional condition that if \(q_{i,v,d,k} > 0\) then \(q_{i,v,d,k} \ge \frac{Q_{\min,k}}{Q_{\text{unit}},k}\).
    \item \(\displaystyle \text{penalty}_{i,d,k} \geq 0\): Unmet demand for cash delivered in denomination \(k\) for ATM \(i\) on day \(d\).
\end{enumerate}

\section*{Objective Function}
We consider a weighted-sum objective function:\\
\textbf{Minimize:}
% \begin{equation*}
%     w_1 \sum_{d \in \mathcal{D}} \sum_{v \in \mathcal{V}} z_{v,d} \;+\; w_2 \sum_{d \in \mathcal{D}} \sum_{i \in \mathcal{I}} I_{i,d} \;+\; w_3 \sum_{d \in \mathcal{D}} \sum_{v \in \mathcal{V}} \sum_{i \in \mathcal{I}} x_{i,v,d} 
%     \;+\; w_4 \sum_{i \in \mathcal{I}}\sum_{d \in \mathcal{D}}\sum_{k \in \mathcal{K}} \text{penalty}_{i,d,k} \;+\; w_5 \sum_{v \in \mathcal{V}}\sum_{d \in \mathcal{D}} \text{cost}_v \cdot z_{v,d}
% \end{equation*}
\begin{align*}
    Z &= w_1\cdot \sum_{d \in \mathcal{D}} \sum_{v \in \mathcal{V}} z_{v,d} \;+\; w_2 \cdot\sum_{d \in \mathcal{D}} \sum_{i \in \mathcal{I}} I_{i,d} \;+\; w_3\cdot \sum_{d \in \mathcal{D}} \sum_{v \in \mathcal{V}} \sum_{i \in \mathcal{I}} x_{i,v,d}\\
     &\;+\; w_4\cdot \sum_{i \in \mathcal{I}}\sum_{d \in \mathcal{D}}\sum_{k \in \mathcal{K}} \text{penalty}_{i,d,k} \;+\; w_5 \cdot\sum_{v \in \mathcal{V}}\sum_{d \in \mathcal{D}} \text{cost}_v \cdot z_{v,d}
\end{align*}

\noindent \textbf{Explanation:}
\begin{itemize}[leftmargin=1cm]
    \item The first term represents the number of vehicles used.
    \item The second term represents the total ATM cash inventory holding cost.
    \item The third term represents the total number of ATM visits.
    \item The fourth term represents the unmet denomination preferences at each of the ATM.
    \item The fifth term represents the cost using the vehicles during the planning horizon.
\end{itemize}

\section*{Constraints}
\subsection*{A. ATM Inventory Balance and Capacity}
\begin{enumerate}[label=\textbf{(A\arabic*)}]
    \item \textbf{Inventory Update:} For each \(i \in \mathcal{I}\) and \(d \in \mathcal{D}\),
    \[
    I_{i,d} =
    \begin{cases}
    I_i^0 + \displaystyle\sum_{v \in \mathcal{V}} y_{i,v,1} - \text{Demand}_{i,1}, & d=1,\\[1mm]
    I_{i,d-1} + \displaystyle\sum_{v \in \mathcal{V}} y_{i,v,d} - \text{Demand}_{i,d}, & d \ge 2.
    \end{cases}
    \]
    \item \textbf{Inventory Limits:} For all \(i \in \mathcal{I}\) and \(d \in \mathcal{D}\),
    \[
        L_i \le I_{i,d}
    \]
    \[
        I_{i,d} \le C_i
    \]
\end{enumerate}

\noindent \textbf{Explanation:}
\begin{itemize}[leftmargin=1cm]
    \item (A1) Keeps track of the cash in the ATM across the planning horizon.
    \item (A2) Ensures the cash in the inventory does not go beyond the lower and upper limits.
\end{itemize}

\subsection*{B. Service (Assignment) Constraints}
\begin{enumerate}[label=\textbf{(B\arabic*)}]
    \item \textbf{One Refill per ATM per Day:} For every \(i \in \mathcal{I}\) and \(d \in \mathcal{D}\),
    \[
    \sum_{v \in \mathcal{V}} x_{i,v,d} \le 1
    \]
    \item \textbf{Linking \(x\) and \(y\):} For all \(i \in \mathcal{I}\), \(v \in \mathcal{V}\), and \(d \in \mathcal{D}\),
    \[
        y_{i,v,d} \ge Q_{\min} \, x_{i,v,d}
    \]
    \[
        y_{i,v,d} \le M\cdot x_{i,v,d}
    \]
\end{enumerate}

\noindent \textbf{Explanation:}
\begin{itemize}[leftmargin=1cm]
    \item (B1) Ensures that each ATM is filled atmost once in a day.
    \item (B2) Ensures that cash is either not deposited or if deposited atleast \(Q_{\min}\) of cash is deposited.
\end{itemize}

\subsection*{C. Vehicle Capacity and Workload}
\begin{enumerate}[label=\textbf{(C\arabic*)}]
    \item \textbf{Vehicle Loading Limit:} For each \(v \in \mathcal{V}\) and \(d \in \mathcal{D}\),
    \[
    \sum_{i \in \mathcal{I}} y_{i,v,d} \le \text{cap}_v \cdot z_{v,d}
    \]
    \item \textbf{Maximum Visits per Vehicle per Day:} For all \(v \in \mathcal{V}\) and \(d \in \mathcal{D}\),
    \[
    \sum_{i \in \mathcal{I}} x_{i,v,d} \le \text{maxVisit}_{v,d} \cdot z_{v,d}
    \]
\end{enumerate}

\noindent \textbf{Explanation:}
\begin{itemize}[leftmargin=1cm]
    \item (C1) Ensures that total cash to be delivered by vehicle \(v\) on a day \(d\) does not exceed its capacity.
    \item (C2) Ensures a vehicle \(v\) does not visit more than \(\text{maxVisit}_{v,d}\) ATMs on day \(d\).
\end{itemize}

\subsection*{D. Security and Scheduling Constraints}
\begin{enumerate}[label=\textbf{(D\arabic*)}]
    \item \textbf{Spread Out ATM Visits:} For all \(v \in \mathcal{V}\), for each \(i \in \mathcal{I}\) and for each day \(d \in \mathcal{D}\) such that \(d+S-1 \le \max(\mathcal{D})\),
    \[
    x_{i,v,d} + \sum_{d' = d+1}^{\min(d+S-1,\max(\mathcal{D}))} x_{i,v,d'} \le 1.
    \]
\end{enumerate}

\noindent \textbf{Explanation:}
\begin{itemize}[leftmargin=1cm]
    \item (D1) Ensures if vehicle \(v\) visits ATM \(i\) on day \(d\), it cannot visit the same ATM again in the next \(S-1\) days.
\end{itemize}

\subsection*{E. Workload Balance Across Vehicles}
\begin{enumerate}[label=\textbf{(E\arabic*)}]
    \item \textbf{Number of ATMs visited:} For each \(v,v' \in \mathcal{V}\),
    % \[
    % W_{v,d} = \sum_{i \in \mathcal{I}} x_{i,v,d}.
    % \]
    % Introduce auxiliary variables \(W_d^{\max}\) and \(W_d^{\min}\) for each day \(d \in \mathcal{D}\) and enforce:
    % \[
    % W_{v,d} \le W_d^{\max} \quad \forall v \in \mathcal{V}, \quad 
    % W_{v,d} \ge W_d^{\min} \quad \forall v \in \mathcal{V},
    % \]
    % \[
    % W_d^{\max} - W_d^{\min} \le \Delta_x \quad \forall d \in \mathcal{D},
    % \]
    % where \(\Delta_x\) is a parameter specifying the maximum allowed difference in the number of ATMs visited by any two vehicles on day \(d\).
    
    \[
    \sum_{i \in \mathcal{I}} \sum_{d \in \mathcal{D}} (x_{i,v,d} - x_{i,v',d}) \le \Delta_x
    \]
    \item \textbf{Cash Deposited in ATMS:} For each \(v,v' \in \mathcal{V}\),
    % \[
    % C_{v,d} = \sum_{i \in \mathcal{I}} y_{i,v,d}.
    % \]
    % Introduce auxiliary variables \(C_d^{\max}\) and \(C_d^{\min}\) for each day \(d \in \mathcal{D}\) and impose:
    % \[
    % C_{v,d} \le C_d^{\max} \quad \forall v \in \mathcal{V}, \quad 
    % C_{v,d} \ge C_d^{\min} \quad \forall v \in \mathcal{V},
    % \]
    % \[
    % C_d^{\max} - C_d^{\min} \le \Delta_y \quad \forall d \in \mathcal{D},
    % \]
    % where \(\Delta_y\) is a parameter specifying the maximum allowed difference in cash deposited by any two vehicles on day \(d\).

    \[
        \sum_{i \in \mathcal{I}} \sum_{d \in \mathcal{D}} (y_{i,v,d} - y_{i,v',d}) \le \Delta_y
    \]
    
\end{enumerate}

\noindent \textbf{Explanation:}
\begin{itemize}[leftmargin=1cm]
    \item (E1) Ensures the difference between the number of ATM visits by any two vehicles \(v\) and \(v'\) across the planning horizon does not exceed \(\Delta_x\).
    \item (E2) Ensures the difference between the total cash deposited by any two vehicles \(v\) and \(v'\) across the planning horizon does not exceed \(\Delta_y\).
\end{itemize}

\subsection*{F. Denomination Constraints (Soft)}
\begin{enumerate}[label=\textbf{(F\arabic*)}]
    \item For each \(i \in \mathcal{I}\), \(d \in \mathcal{D}\), and \(k \in \mathcal{K}\),
    \[
    \sum_{v \in \mathcal{V}} y_{i,v,d,k} + \text{penalty}_{i,d,k} \ge r_{i,d,k} 
    \]
    \item For each \(i \in \mathcal{I}\), \(d \in \mathcal{D}\)
    \[
    \sum_{k \in \mathcal{K}}y_{i,v,d,k} = y_{i,v,d}
    \]
    % \item For each \(k \in \mathcal{K}, i \in \mathcal{I}, v \in \mathcal{V}\) and \(d \in \mathcal{D}\)
    % \[
    % x_{i,v,d,k} \le x_{i,v,d}
    % \] 
\end{enumerate}
\noindent \textbf{Explanation:}
\begin{itemize}[leftmargin=1cm]
    \item (F1) Penalizes unmet denomination demands for denomination \(k\) on day \(d\) at ATM \(i\).
    \item (F2) Ensures total cash deposited is the sum of the total cash deposited in each denomination.
\end{itemize}

\subsection*{G. Deposit Size Discreteness}
\begin{enumerate}[label=\textbf{(G\arabic*)}]
    \item To ensure that if cash is either not delivered or at least \(Q_{\min}\) (in multiples of \(Q_{\text{unit}}\)):
    \[
        y_{i,v,d} = Q_{\text{unit}} \cdot q_{i,v,d}
    \]
    \[
        q_{i,v,d} \ge \frac{Q_{\min}}{Q_{\text{unit}}}\cdot x_{i,v,d}
    \]
    \item To ensure that cash in denomination \(k\) is delivered, is in multiples of \(Q_{\text{unit},k}\):
    \[
        y_{i,v,d,k} = Q_{\text{unit},k} \cdot q_{i,v,d,k}
    \]
    % \[
    %     q_{i,v,d,k} \ge \frac{Q_{\min,k}}{Q_{\text{unit},k}}\cdot x_{i,v,d,k}
    % \]
\end{enumerate}

\noindent \textbf{Explanation:}
\begin{itemize}[leftmargin=1cm]
    \item (G1) Ensures that for ATM \(i\), cash is either not deposited by a vehicle \(v\) on day \(d\) or atleast \(Q_{\min}\) is deposited in multiples of \(Q_{\text{unit}}\)
    \item (G2) Ensures that cash in each denomination \(k\) is deposited in non-zero multiples of \(Q_{\min,k}\).
\end{itemize}

\section*{Assumptions}
\begin{itemize}[leftmargin=1.5cm]
    \item The planning horizon is one week (\(7\) days).
    \item Daily demands \(\text{Demand}_{i,d}\) and initial inventories \(I_i^0\) are known.
    \item ATMs must always have cash between the levels \(L_i\) and \(C_i\).
    \item Vehicles are loaded only at the beginning of the day (no mid-day refilling).
    \item Vehicles' cash capacities are limited by security considerations, which may be lower than their physical capacities.
    \item In case of high demand, additional (rented) vehicles may be used, represented by higher cost parameters.
    \item A refill at an ATM is done by at most one vehicle per day.
    \item For security, if a vehicle visits an ATM on day \(d\), it cannot visit the same ATM again for the next \(S-1\) days.
    \item Refills must be in multiples of \(Q_{\text{unit}}\) (e.g., 10K), with any nonzero refill being at least \(Q_{\min}\) (e.g., 50K).
    \item The denomination requirement is a soft constraint; unmet denomination preferences are penalized but do not prevent meeting the overall cash demand.
\end{itemize}

\section*{Output}
The solution to this Mixed-Integer Programming (MIP) model will determine:
\begin{itemize}[leftmargin=1.5cm]
    \item Which vehicles are used on each day (via \(z_{v,d}\)).
    \item Which ATMs are refilled by which vehicle on each day (via \(x_{i,v,d}\)).
    \item The amount of cash delivered to each ATM on each day (via \(y_{i,v,d}\)).
    \item The amount of cash in a denomination delivered to each ATM on each day (via \(y_{i,v,d,k}\)).
    \item The evolution of ATM cash inventories \(I_{i,d}\) over the planning horizon.
\end{itemize}

\end{document}
