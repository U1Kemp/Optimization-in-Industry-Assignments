\documentclass[11pt]{article}
\usepackage{amsmath,amssymb,amsthm,amsfonts}
\usepackage{geometry}
\usepackage{enumitem}
\usepackage{setspace}
\geometry{margin=1in}
\onehalfspacing

\begin{document}

\title{Mathematical Model for Optimal ATM Refill Problem}
\author{}
\date{}
\maketitle

\section{Sets}
\begin{itemize}[leftmargin=1.5cm]
    \item[\textbf{I.}] \(\mathcal{I}\): Set of ATMs, indexed by \(i\) (i.e., \(i=1,\dots,|\mathcal{I}|\)).
    \item[\textbf{V.}] \(\mathcal{V}\): Set of vehicles, indexed by \(v\) (i.e., \(v=1,\dots,|\mathcal{V}|\)).
    \item[\textbf{D.}] \(\mathcal{D}\): Set of days in the planning horizon, indexed by \(d\) (e.g., \(d=1,\dots,7\)).
    \item[\textbf{K.}] \(\mathcal{K}\): (Optional) Set of denominations, indexed by \(k\).
\end{itemize}

\section{Parameters}
\begin{itemize}[leftmargin=1.5cm]
    \item \(\displaystyle d_{i,d}\): Cash demand at ATM \(i\) on day \(d\).
    \item \(\displaystyle I_i^0\): Initial cash inventory at ATM \(i\) (at start of day 1).
    \item \(\displaystyle C_i\): Maximum cash capacity of ATM \(i\).
    \item \(\displaystyle L_i\): Minimum cash level required at ATM \(i\).
    \item \(\displaystyle Q_{\min}\): Minimum cash deposit per refill (e.g., 50K).
    \item \(\displaystyle Q_{\text{unit}}\): Cash deposit unit (e.g., 10K), so that any nonzero refill is a multiple of \(Q_{\text{unit}}\).
    \item \(\displaystyle \text{cap}_v\): Maximum cash carrying capacity of vehicle \(v\) (security limited).
    \item \(\displaystyle \text{cost}_v\): Cost associated with using vehicle \(v\).
    \item \(\displaystyle \text{maxVisit}_{v,d}\): Maximum number of ATMs vehicle \(v\) can service on day \(d\) (e.g., 20 on weekdays, 30 on weekends).
    \item \(\displaystyle S\): Security gap in days. If vehicle \(v\) visits ATM \(i\) on day \(d\), it cannot visit the same ATM again in the next \(S-1\) days.
    \item \(\displaystyle r_{i,d,k}\): (Optional) Denomination requirement for ATM \(i\) on day \(d\) for denomination \(k\) (soft constraint).
    \item \(\displaystyle w_1,\, w_2,\, w_3\): Weights for the objective function corresponding to vehicles used, ATM inventory holding cost, and number of visits, respectively.
    \item \(\displaystyle M\): A sufficiently large constant (Big-M).
\end{itemize}

\section{Decision Variables}
\begin{itemize}[leftmargin=1.5cm]
    \item \(\displaystyle x_{i,v,d} \in \{0,1\}\): Equals 1 if vehicle \(v\) refills ATM \(i\) on day \(d\); 0 otherwise.
    \item \(\displaystyle y_{i,v,d} \ge 0\): Amount of cash delivered to ATM \(i\) by vehicle \(v\) on day \(d\).  
    \[
    y_{i,v,d} = Q_{\text{unit}} \cdot q_{i,v,d}, \quad q_{i,v,d} \in \mathbb{Z}_{\geq 0}
    \]
    with the additional condition that if \(q_{i,v,d} > 0\) then \(q_{i,v,d} \ge \frac{Q_{\min}}{Q_{\text{unit}}}\).
    \item \(\displaystyle z_{v,d} \in \{0,1\}\): Equals 1 if vehicle \(v\) is used on day \(d\); 0 otherwise.
    \item \(\displaystyle I_{i,d}\): Cash inventory at ATM \(i\) at the end of day \(d\).
    \item \(\displaystyle y_{i,v,d,k} \ge 0\) (Optional): Cash delivered in denomination \(k\) for ATM \(i\) on day \(d\).
\end{itemize}

\section{Objective Function}
We consider a weighted-sum objective that minimizes:
\begin{equation*}
    \min \quad Z = w_1 \sum_{d \in \mathcal{D}} \sum_{v \in \mathcal{V}} z_{v,d} \;+\; w_2 \sum_{d \in \mathcal{D}} \sum_{i \in \mathcal{I}} I_{i,d} \;+\; w_3 \sum_{d \in \mathcal{D}} \sum_{v \in \mathcal{V}} \sum_{i \in \mathcal{I}} x_{i,v,d}.
\end{equation*}

\noindent \textbf{Explanation:}
\begin{itemize}[leftmargin=1cm]
    \item The first term minimizes the (weighted) number of vehicles used.
    \item The second term minimizes the total ATM cash inventory holding cost.
    \item The third term minimizes the total number of ATM visits.
\end{itemize}

\section{Constraints}
\subsection*{A. ATM Inventory Balance and Capacity}
\begin{enumerate}[label=\textbf{(A\arabic*)}]
    \item \textbf{Inventory Update:} For each \(i \in \mathcal{I}\) and \(d \in \mathcal{D}\),
    \[
    I_{i,d} =
    \begin{cases}
    I_i^0 + \displaystyle\sum_{v \in \mathcal{V}} y_{i,v,1} - d_{i,1}, & d=1,\\[1mm]
    I_{i,d-1} + \displaystyle\sum_{v \in \mathcal{V}} y_{i,v,d} - d_{i,d}, & d \ge 2.
    \end{cases}
    \]
    \item \textbf{Inventory Limits:} For all \(i \in \mathcal{I}\) and \(d \in \mathcal{D}\),
    \[
    L_i \le I_{i,d} \le C_i.
    \]
\end{enumerate}

\subsection*{B. Service (Assignment) Constraints}
\begin{enumerate}[label=\textbf{(B\arabic*)}]
    \item \textbf{One Refill per ATM per Day:} For every \(i \in \mathcal{I}\) and \(d \in \mathcal{D}\),
    \[
    \sum_{v \in \mathcal{V}} x_{i,v,d} \le 1.
    \]
    \item \textbf{Linking \(x\) and \(y\):} For all \(i \in \mathcal{I}\), \(v \in \mathcal{V}\), and \(d \in \mathcal{D}\),
    \[
    y_{i,v,d} \ge Q_{\min} \, x_{i,v,d} \quad \text{and} \quad y_{i,v,d} \le M\, x_{i,v,d}.
    \]
\end{enumerate}

\subsection*{C. Vehicle Capacity and Workload}
\begin{enumerate}[label=\textbf{(C\arabic*)}]
    \item \textbf{Vehicle Loading Limit:} For each \(v \in \mathcal{V}\) and \(d \in \mathcal{D}\),
    \[
    \sum_{i \in \mathcal{I}} y_{i,v,d} \le \text{cap}_v \, z_{v,d}.
    \]
    \item \textbf{Maximum Visits per Vehicle per Day:} For all \(v \in \mathcal{V}\) and \(d \in \mathcal{D}\),
    \[
    \sum_{i \in \mathcal{I}} x_{i,v,d} \le \text{maxVisit}_{v,d} \, z_{v,d}.
    \]
\end{enumerate}

\subsection*{D. Security and Scheduling Constraints}
\begin{enumerate}[label=\textbf{(D\arabic*)}]
    \item \textbf{Spread Out ATM Visits:} For all \(v \in \mathcal{V}\), for each \(i \in \mathcal{I}\) and for each day \(d \in \mathcal{D}\) such that \(d+S-1 \le \max(\mathcal{D})\),
    \[
    x_{i,v,d} + \sum_{d' = d+1}^{\min(d+S-1,\max(\mathcal{D}))} x_{i,v,d'} \le 1.
    \]
\end{enumerate}

\subsection*{E. (Optional) Denomination Constraints (Soft)}
\begin{enumerate}[label=\textbf{(E\arabic*)}]
    \item For each \(i \in \mathcal{I}\), \(d \in \mathcal{D}\), and \(k \in \mathcal{K}\), if modeling denomination details,
    \[
    \sum_{v \in \mathcal{V}} y_{i,v,d,k} \ge r_{i,d,k} - \text{penalty}_{i,d,k}.
    \]
    A penalty term may be added to the objective to account for unmet denomination preferences.
\end{enumerate}

\subsection*{F. Deposit Size Discreteness}
\begin{enumerate}[label=\textbf{(F\arabic*)}]
    \item To ensure that if cash is delivered, it is either zero or at least \(Q_{\min}\) (in multiples of \(Q_{\text{unit}}\)):
    \[
    y_{i,v,d} = Q_{\text{unit}} \cdot q_{i,v,d}, \quad q_{i,v,d} \in \mathbb{Z}_{\geq 0}, \quad q_{i,v,d} \ge \frac{Q_{\min}}{Q_{\text{unit}}}\, x_{i,v,d}.
    \]
\end{enumerate}

\section{Assumptions}
\begin{itemize}[leftmargin=1.5cm]
    \item The planning horizon is one week (\(7\) days).
    \item Daily demands \(d_{i,d}\) and initial inventories \(I_i^0\) are known.
    \item ATMs must always have cash between the levels \(L_i\) and \(C_i\).
    \item Vehicles are loaded only at the beginning of the day (no mid-day refilling).
    \item Vehicles' cash capacities are limited by security considerations, which may be lower than their physical capacities.
    \item In case of high demand, additional (rented) vehicles may be used, represented by higher cost parameters.
    \item A refill at an ATM is done by at most one vehicle per day.
    \item For security, if a vehicle visits an ATM on day \(d\), it cannot visit the same ATM again for the next \(S-1\) days.
    \item Refills must be in multiples of \(Q_{\text{unit}}\) (e.g., 10K), with any nonzero refill being at least \(Q_{\min}\) (e.g., 50K).
    \item The denomination requirement is a soft constraint; unmet denomination preferences are penalized but do not prevent meeting the overall cash demand.
\end{itemize}

\section{Output}
The solution to this Mixed-Integer Programming (MIP) model will determine:
\begin{itemize}[leftmargin=1.5cm]
    \item Which vehicles are used on each day (via \(z_{v,d}\)).
    \item Which ATMs are refilled by which vehicle on each day (via \(x_{i,v,d}\)).
    \item The amount of cash delivered to each ATM on each day (via \(y_{i,v,d}\)).
    \item The evolution of ATM cash inventories \(I_{i,d}\) over the planning horizon.
\end{itemize}

\end{document}
