\documentclass[11pt]{article}
\usepackage{amsmath, amssymb}
\usepackage{geometry}
\geometry{letterpaper, margin=1in}

\begin{document}

\title{Optimization Assignment 4 \\School Timetabling Optimization Problem}
\author{Utpalraj Kemprai}
\date{}
\maketitle

\section*{Sets}
\begin{itemize}
    \item $S$: Set of subjects, $S = \{1,2,\cdots, 10\}$
    \item $C$: Set of sections, $C = \{1,2,3,4\}$
    \item $D$: Set of days, $D = \{1,2,3,4,5\}$
    \item $T$: Timeslots of a day, $T = \{1,2,\cdots,8\}$
    \item $S_{\text{sci}}$: Science subjects
    \item $S_{\text{nonsci}}$: Non-science subjects
    \item $T_{\text{morning}}$ = \{1,2,3,4\}
    \item $T_{\text{afternoon}}$ = \{5,6,7,8\}
\end{itemize}

\section*{Parameter}
\begin{itemize}
    \item \(M\) = 5 (= no. of sections + 1): A sufficiently large constant (big-M) used for linearization. 
\end{itemize}

\section*{Decision Variable}
\begin{itemize}
    \item $x_{s,c,d,t}$: Binary variable, 1 if subject \(s\) is assigned to section \(c\) on day \(d\) at timeslot \(t\), 0 otherwise
    \item $p_{s,d,t}$: Binary variable, 1 if practical session for subject \(s\) is assigned on day \(d\) at timeslot \(t\), 0 otherwise
    \item $y_{c,d,t}^{\text{sci}}$: Binary variable, 1 if a science subject is assigned to section \(c\) on day \(d\) at times \(t\), 0 otherwise
    \item $y_{c,d,t}^{\text{nonsci}}$: Binary variable, 1 if a non-science subject is assigned to section \(c\) on day \(d\) at times \(t\), 0 otherwise
    \item $z_{c,d,t}^{\text{sci}} (t \in \{1,2,\cdots,7\})$: Binary variable, 1 if consecutive science subjects were assigned to section \(c\) on day \(d\) in timeslots \(t\) and \(t+1\), otherwise 0
    \item $z_{c,d,t}^{\text{nonsci}} (t \in \{1,2,\cdots,7\})$: Binary variable, 1 if consecutive science subjects were assigned to section \(c\) on day \(d\) in timeslots \(t\) and \(t+1\), otherwise 0
    \item $n_{s,t}$: Number of times subject \(s\) is assigned to timeslot \(t\) in the week
    \item $u_{s}$: Maximum number of times subject \(s\) is assigned to a single timeslot
\end{itemize}
\textbf{Explanation:} The \(x\)-variables represent the assignment of subjects to timeslots across sections. The \(p\)-variables represent the allocation of practicals for the subjects. The \(y\)-variables represents if the allocated subjects are science or non-science. The \(y\) and \(z\) variables are used to penalize consecutive science and non-science allocation. And finally the \(n\) and \(u\) variables are used to promote the diversity of timeslots for the subjects.

\section*{Objective Function}
We want to :
\begin{enumerate}
    \item Minimize the number of practicals before lunch
    \item Minimize the number of consecutive science and non-science classes
    \item Diversify the time slots allocated for a subject
\end{enumerate}

\noindent A weighted objective function is given by
\[
\min \left( 
\alpha_1 \sum_{s \in S}\sum_{d \in D}\sum_{t \in T_\text{morning}} p_{s,d,t} + \alpha_2 \sum_{s \in S}\sum_{d \in D}\sum_{t=1}^{7}\sum_{c \in C} \left(z_{c,d,t}^{\text{sci}} + z_{c,d,t}^{\text{nonsci}}\right)
+ \alpha_3 \sum_{s \in S} u_s
\right)
\]
where \(\alpha_1,\alpha_2 \text{ and } \alpha_3\) are positive weights reflecting the importance of each objective.

\textbf{Explanation:} The first term penalizes the allocation of practicals before lunch, the second term penalizes consecutive science and non-science classes and the third term penalizes the number of times a subject is allocated to a particular time slots across the days of the week.

\section*{Constraints}
\begin{enumerate}
    \item Each subject has 4 classes in a week:
    \[
    \sum_{d \in D} \sum_{t \in T} x_{s,c,d,t} = 4 \quad \forall s \in \mathcal{S}, \forall c \in \mathcal{C}
    \]
    \textbf{Explanation:} Ensures each subject has 4 classes in a week.

    \item Every science subject to have one practical in a week:
    \[
    \sum_{d \in D}\sum_{t \in T} p_{s,d,t} = 1 \quad \forall s \in S_{\text{sci}}
    \]
    \textbf{Explanation:} Ensures each science subject has exactly one practical in a week.

    \item Only science subjects have practicals
    \[
    p_{s,d,t} = 0 \quad \forall s \in S_{\text{nonsci}}
    \]
    \textbf{Explanation:} Ensures that non-science have no practicals.
    
    \item At most one class per subject per day:
    \[
    \sum_{t \in T} x_{s,c,d,t} \leq 1 \quad \forall s \in S, \forall c \in C, \forall d \in D
    \]
    \textbf{Explanation:} Each subject has atmost one class in a day for all the sections.

    \item Atmost one practical per section per day:
    \[
    \sum_{s \in \mathcal{S}_\text{sci}} \sum_{t \in \mathcal{T}} p_{s,d,t} \leq 1 \quad \forall c \in C, \forall d \in D
    \]
    \textbf{Explanation:} Ensures that we have atmost one practical session in a day.

    % \item Prefer practicals after lunch (soft):
    % \[
    % \sum_{s \in \mathcal{S}_\text{sci}} \sum_{c \in \mathcal{C}} \sum_{d \in \mathcal{D}} \sum_{t \in \mathcal{T}_\text{morning}} x_{s,c,d,t}^{\text{practical}} \quad \text{minimize}
    % \]
    
    \item Practical session should take place in one of the scheduled class
    \[
        p_{s,d,t} \leq x_{s,c,d,t} \quad \forall s \in S, c \in C, d \in D, t \in T
    \]
    \textbf{Explanation:} Ensures practical session for a subject is scheduled in one of the timeslots where that subject is already scheduled.  

    \item Atmost one subject in a timeslot for a section
    \[
        \sum_{s \in S} x_{s,c,d,t} \leq 1 \quad \forall c \in C, d \in D, t \in T
    \]
    \textbf{Explanation:} Ensures that no overlap of subjects occurs for a timeslot for any Section.

    \item Sections don't have same subject class simultaneously unless practical
    \[
        \sum_{c \in C} x_{s,c,d,t} \geq 4 \times p_{s,d,t} \quad \forall s \in S, d \in D, t \in T
    \]
    \[
        \sum_{c \in C} x_{s,c,d,t} \leq 1 + M \times p_{s,d,t} \quad \forall s \in S, d \in D, t \in T
    \]
    \textbf{Explanation:}
    \begin{itemize}
        \item The first constraint ensures that if a practical for a subject is scheduled at a particular timeslot, it is scheduled for all the sections.
        \item The second constraint ensures that if no practical for a subject is scheduled, atmost one of the sections can have a class of that subject.
    \end{itemize}

    \item Consecutive science / non-science penalty
    \begin{align*}
        y^{\mathrm{sci}}_{c,d,t} &\;=\; \sum_{s\in S_{\mathrm{sci}}} x_{s,c,d,t} \quad \forall c \in C, t \in T, d \in D\\
        y^{\mathrm{nonsci}}_{c,d,t} &\;=\; \sum_{s\in S_{\mathrm{nonsci}}} x_{s,c,d,t} \quad \forall c \in C, t \in T, d \in D\\
        z^{\mathrm{sci}}_{c,d,t} &\ge y^{\mathrm{sci}}_{c,d,t}+y^{\mathrm{sci}}_{c,d,t+1}-1 \quad \forall c \in C, t \in T, d \in D\\
        z^{\mathrm{nonsci}}_{c,d,t} &\ge y^{\mathrm{nonsci}}_{c,d,t}+y^{\mathrm{nonsci}}_{c,d,t+1}-1 \quad \forall c \in C, t \in T, d \in D
      \end{align*}
      \textbf{Explanation:}
      \begin{itemize}
        \item The first two constraints ensures that \(y^{\mathrm{sci}}_{c,d,t}\) (or \(y^{\mathrm{nonsci}}_{c,d,t}\)) is 1 if a science (or non-science) is scheduled for section \(c\) on day \(d\) for timeslot \(t\) and 0 otherwise.
        \item The last two constraints ensures that \(z^{\mathrm{sci}}_{c,d,t}\) (or \(z^{\mathrm{nonsci}}_{c,d,t}\)) is 1 if two consecutive science (or non-science) classes are scheduled for section \(c\) on day \(d\) for timeslots \(t\) and \(t+1\). Otherwise,  \(z^{\mathrm{sci}}_{c,d,t}\) (or \(z^{\mathrm{nonsci}}_{c,d,t}\)) is 0.
      \end{itemize}

      \item Timeslot diversity penalty
      \begin{align*}
        n_{s,t} &\;=\; \sum_{c\in C}\sum_{d\in D} x_{s,c,d,t}\\
        u_s &\;\geq\; n_{s,t} \quad \forall s \in S, t \in T
      \end{align*}
      \textbf{Explanation:}
      \begin{itemize}
        \item The first constraint ensures that \(n_{s,t}\) is equal to the number of times subject \(s\) is scheduled for timeslot \(t\) in the week.
        \item The second constraint ensures that \(u_s\) is equal to \(\max\limits_{t \in T}n_{s,t}\).
      \end{itemize}
      
\end{enumerate}
\section*{Necessary Assumptions}
\begin{itemize}
    \item Every subject requires exactly 4 sessions per week, with science subjects including exactly one practical.
    \item Lunch break perfectly divides the day into morning and afternoon blocks.
    \item All sections have identical timetabling requirements and must satisfy the same constraints.
    \item Soft penalties (consecutive runs, timeslot diversity) are weighted and balanced via \(\alpha_i\)'s chosen by the planner.
\end{itemize}

\end{document}
