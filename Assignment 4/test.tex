\documentclass[11pt]{article}
\usepackage{amsmath, amssymb}
\usepackage{geometry}
\geometry{a4paper, margin=1in}

\begin{document}

\title{Optimization Assignment 4 \\School Timetabling Optimization Problem}
\author{Utpalraj Kemprai}
\date{}
\maketitle

\section*{Sets}
\begin{itemize}
    \item $S$: Set of subjects, $S = \{1,2,\cdots, 10\}$
    \item $C$: Set of sections, $C = \{1,2,3,4\}$
    \item $D$: Set of days, $D = \{1,2,3,4,5\}$
    \item $T$: Timeslots of a day, $T = \{1,2,\cdots,8\}$
    \item $S_{\text{sci}}$: Science subjects
    \item $S_{\text{nonsci}}$: Non-science subjects
    \item $T_{\text{morning}}$ = \{1,2,3,4\}
    \item $T_{\text{afternoon}}$ = \{5,6,7,8\}
\end{itemize}

\section*{Decision Variable}
\begin{itemize}
    \item $x_{s,c,d,t}$: Binary variable, 1 if subject \(s\) is assigned to section \(c\) on day \(d\) at timeslot \(t\), 0 otherwise.
    \item $p_{s,c,d,t}$: Binary variable, 1 if practical session for subject \(s\) is assigned to section \(c\) on day \(d\) at timeslot \(t\), 0 otherwise.
    \item $y_{c,d,t}^{\text{sci}}$: Binary variable, 1 if a science subject is assigned to section \(c\) on day \(d\) at times \(t\), 0 otherwise.
    \item $y_{c,d,t}^{\text{nonsci}}$: Binary variable, 1 if a non-science subject is assigned to section \(c\) on day \(d\) at times \(t\), 0 otherwise.
    \item $z_{c,d,t}^{\text{sci}} (t \in \{1,2,\cdots,7\})$: Binary variable, 1 if consecutive science subjects were assigned to section \(c\) on day \(d\) in timeslots \(t\) and \(t+1\), otherwise 0.
    \item $z_{c,d,t}^{\text{nonsci}} (t \in \{1,2,\cdots,7\})$: Binary variable, 1 if consecutive science subjects were assigned to section \(c\) on day \(d\) in timeslots \(t\) and \(t+1\), otherwise 0.
    \item $n_{s,t}$: Number of times subject \(s\) is assigned to timeslot \(t\) in the week.
    \item $u_{s}$: Maximum number of times subject \(s\) is assigned to a single timeslot.
\end{itemize}
\textbf{Explanation:} The \(x\)-variables represent the assignment of subjects to timeslots across sections. The \(p\)-variables represent the allocation of practicals for the subjects. The \(y\)-variables represents if the allocated subjects are science or non-science. The \(z\) variables are used to penalize consecutive science and non-science allocation. And finally the \(n\) and \(u\) variables are used to promote the diversity of timeslots for the subjects.

\section*{Objective Function}
We want to create an optimal timetable by minimizing:
\begin{enumerate}
    \item Number of practicals scheduled before lunch.
    \item Number of consecutive science/non-science classes.
    \item Repetitive scheduling of subjects at the same timeslot.
\end{enumerate}

\noindent A weighted objective function is given by,
\[
\min \left( 
\alpha_1 \sum_{s \in S}\sum_{d \in D}\sum_{c \in C}\sum_{t \in T_\text{morning}} p_{s,c,d,t} + \alpha_2 \sum_{s \in S}\sum_{d \in D}\sum_{t=1}^{7}\sum_{c \in C} \left(z_{c,d,t}^{\text{sci}} + z_{c,d,t}^{\text{nonsci}}\right)
+ \alpha_3 \sum_{s \in S} u_s
\right)
\]
\indent where \(\alpha_1,\alpha_2 \text{ and } \alpha_3\) are positive weights reflecting the importance of each objective.

\noindent\textbf{Explanation:} The first term penalizes the allocation of practicals before lunch, the second term penalizes consecutive science and non-science classes and the third term penalizes the number of times a subject is allocated to a single time slot across the days of the week.

\section*{Constraints}
\begin{enumerate}
    \item Weekly Class Requirements
    \[
    \sum_{d \in D} \sum_{t \in T} x_{s,c,d,t} = 4 \quad \forall s \in S, \forall c \in C
    \]
    \textbf{Explanation:} 

    Each subject must have exactly 4 classes per week for each section.

    \item Science Subjects' Practical Requirements
    \[
    \sum_{d \in D}\sum_{t \in T} p_{s,c,d,t} = 1 \quad \forall s \in S_{\text{sci}}
    \]
    \textbf{Explanation:} 

    Ensures each science subject has exactly one practical session per week for each section.

    \item Non-Science Subjects Have No Practicals
    \[
    p_{s,c,d,t} = 0 \quad \forall s \in S_{\text{nonsci}}, c \in C, d \in D, t \in T
    \]
    \textbf{Explanation:} 

    Ensures that non-science have no practicals.
    
    \item At most one class per subject per day for each section:
    \[
    \sum_{t \in T} x_{s,c,d,t} \leq 1 \quad \forall s \in S, \forall c \in C, \forall d \in D
    \]
    \textbf{Explanation:} 

    Ensure that each subject has atmost one class in a day for all the sections.

    \item Atmost one practical per section per day:
    \[
    \sum_{s \in S_\text{sci}} \sum_{t \in T} p_{s,c,d,t} \leq 1 \quad \forall c \in C, \forall d \in D
    \]
    \textbf{Explanation:} 

    Ensures that each section has atmost one practical session in a day.
    
    \item Practical Sessions Must Be Scheduled During Regular Classes
    \[
        p_{s,c,d,t} \leq x_{s,c,d,t} \quad \forall s \in S, c \in C, d \in D, t \in T
    \]
    \textbf{Explanation:} 

    Ensures that practical session of a subject is scheduled in one of the regular classes for each section. 

    \item No overlapping subjects
    \[
        \sum_{s \in S} x_{s,c,d,t} \leq 1 \quad \forall c \in C, d \in D, t \in T
    \]
    \textbf{Explanation:} 

    Ensures that each section can have atmost one subject in any timeslot.

    \item Laboratory Capacity Constraints
    \[
        \sum_{c \in C} p_{s,c,d,t} \leq 1  \quad \forall s \in S_{\text{sci}}, d \in D, t \in T
    \]

    \textbf{Explanation:}

    Since there's only one practical laboratory per science subject, no more than one section can have a practical for the same subject at the same time.

    \item No consecutive practical classes
    \[
    \sum_{s \in S_{\text{sci}}} (p_{s,c,d,t} + p_{s,c,d,t+1}) \leq 1 \quad \forall c \in C, d \in D, t \in \{1,2,\cdots,7\}
    \]

    \textbf{Explanation:}

    Ensures that no section has consecutive practical sessions.

    \item Track consecutive science and non-science classes
    \begin{align*}
        y^{\mathrm{sci}}_{c,d,t} &\;=\; \sum_{s\in S_{\mathrm{sci}}} x_{s,c,d,t} \quad \forall c \in C, t \in T, d \in D\\
        y^{\mathrm{nonsci}}_{c,d,t} &\;=\; \sum_{s\in S_{\mathrm{nonsci}}} x_{s,c,d,t} \quad \forall c \in C, t \in T, d \in D\\
        z^{\mathrm{sci}}_{c,d,t} &\ge y^{\mathrm{sci}}_{c,d,t}+y^{\mathrm{sci}}_{c,d,t+1}-1 \quad \forall c \in C, t \in T, d \in D\\
        z^{\mathrm{nonsci}}_{c,d,t} &\ge y^{\mathrm{nonsci}}_{c,d,t}+y^{\mathrm{nonsci}}_{c,d,t+1}-1 \quad \forall c \in C, t \in T, d \in D
    \end{align*}
      \textbf{Explanation:}
      \begin{itemize}
        \item The first two constraints ensures that \(y^{\mathrm{sci}}_{c,d,t}\) (or \(y^{\mathrm{nonsci}}_{c,d,t}\)) is 1 if a science (or non-science) is scheduled for section \(c\) on day \(d\) for timeslot \(t\) and 0 otherwise.\\\\
        \(y^{\mathrm{sci}}_{c,d,t}\) and \(y^{\mathrm{nonsci}}_{c,d,t}\) basically track which type of subject is scheduled.

        \item The last two constraints ensures that \(z^{\mathrm{sci}}_{c,d,t}\) (or \(z^{\mathrm{nonsci}}_{c,d,t}\)) is 1 if two consecutive science (or non-science) classes are scheduled for section \(c\) on day \(d\) for timeslots \(t\) and \(t+1\). Otherwise,  \(z^{\mathrm{sci}}_{c,d,t}\) (or \(z^{\mathrm{nonsci}}_{c,d,t}\)) is 0.\\\\
        \(z^{\mathrm{sci}}_{c,d,t}\) and \(z^{\mathrm{nonsci}}_{c,d,t}\) basically identify when two consecutive periods have the same type of subjects for a section.
      \end{itemize}

      \item Tracking timeslot usage
      \begin{align*}
        n_{s,t} &\;=\; \sum_{c\in C}\sum_{d\in D} x_{s,c,d,t}\\
        u_s &\;\geq\; n_{s,t} \quad \forall s \in S, t \in T
      \end{align*}
      \textbf{Explanation:}
      \begin{itemize}
        \item The first constraint ensures that \(n_{s,t}\) is equal to the number of times subject \(s\) is scheduled for timeslot \(t\) in the week.
        \item The second constraint ensures that \(u_s\) is equal to \(\max\limits_{t \in T}n_{s,t}\) i.e. the maximum number of times subject \(s\) is scheduled in a single timeslot.
      \end{itemize}
      
\end{enumerate}
\section*{Necessary Assumptions}
\begin{itemize}
    \item Each subject requires exactly 4 sessions per week.
    \item Science subjects require exactly 1 practical session out of their 4 weekly sessions.
    \item Lunch break occurs between timeslots 4 and 5.
    \item All sections have identical timetabling requirements and must satisfy the same constraints.
    \item All constraints are equally applicable across all sections.
    \item The weights in the objective function \((\alpha_1,\alpha_2,\alpha_3)\) can be adjusted to prioritize different aspects of the schedule.
\end{itemize}

\end{document}
