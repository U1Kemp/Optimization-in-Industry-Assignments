\documentclass[11pt]{article}
\usepackage{amsmath, amssymb}
\usepackage{geometry}
\geometry{letterpaper, margin=1in}
\usepackage{enumitem}

\begin{document}

\title{Optimization Assignment 3 \\Model for Office Space Planning}
\author{Utpalraj Kemprai}
\date{}
\maketitle

\section{Sets}
\begin{itemize}[leftmargin=2em]
    \item[$\bullet$] \textbf{$I$}: Set of teams, indexed by $i$.
    \item[$\bullet$] \textbf{$J$}: Set of days in the planning horizon, indexed by $j$.
    \item[$\bullet$] \textbf{$K$}: Set of floors (or zones) in the office, indexed by $k$.
\end{itemize}

\section{Parameters}
\begin{itemize}[leftmargin=2em]
    \item[$\bullet$] $E_i$: Total number of employees in team $i$.
    \item[$\bullet$] $F_k$: Capacity of floor $k$ (adjusted for social distancing).
    \item[$\bullet$] $S_{ij}$: Number of fixed seats (assured allocation) for team $i$ on day $j$.
    \item[$\bullet$] $T_{ij}$: Minimum threshold (from the collaboration map) for team $i$ on day $j$.
    \item[$\bullet$] $P \in [0,100]$: Allowed percentage (e.g., 20\%) for the difference between the maximum and minimum daily occupancy.
    \item[$\bullet$] \(M = \left(1 + \max_{i \in I} E_{i}\right)\): A sufficiently large constant (big-M) used for linearization.
\end{itemize}
\textbf{Explanation:} These parameters supply the required input data (team sizes, floor capacities, and collaboration targets). Note that $T_{ij}$ represents the target number of employees that should be present for effective collaboration on day $j$ for team $i$.

\section{Decision Variables}
\begin{itemize}[leftmargin=2em]
    \item[$\bullet$] $x_{ijk}$: Number of employees from team $i$ assigned to floor $k$ on day $j$. \\
    \quad (Assumed to be integer and $x_{ijk} \ge 0$.)
    \item[$\bullet$] $y_{ijk}$: Binary variable equal to 1 if team $i$ occupies floor $k$ on day $j$ (i.e., if $x_{ijk} > 0$), and 0 otherwise.
    \item[$\bullet$] $z_{ik}$: Binary variable equal to 1 if team $i$ ever occupies floor $k$ (across all days), and 0 otherwise.
    \item[$\bullet$] $w_{jk}$: Binary variable equal to 1 if floor $k$ is occupied by any of the teams on day $j$, and 0 otherwise.
    \item[$\bullet$] $L_j$: Total number of employees allocated on day $j$, defined as
    \[
    L_j = \sum_{i \in I} \sum_{k \in K} x_{ijk}.
    \]
    \item[$\bullet$] $L_{\min}$: A variable representing the minimum daily total occupancy over all days.
    \item[$\bullet$] $u_{ij}$: Binary variable equal to 1 if the collaboration target for team $i$ on day $j$ is shifted to another day, and 0 if not.
\end{itemize}
\textbf{Explanation:} The $x$-variables represent the primary allocation. The $y$- and $z$-variables help ensure minimum allocations and control multi-floor usage. The new variable $u_{ij}$ enables the model to “relax” the collaboration target on day $j$ for team $i$ (i.e., to shift that day’s collaboration if needed). Finally, $L_j$ and $L_{\min}$ support the fairness constraints.

\section{Objective Function}
We aim to:
\begin{enumerate}
    \item Maximize total occupancy.
    \item Ensure fair share allocation across days.
    \item Minimize the number of floors used by each team.
    \item Minimize the number of teams occupying multiple floors.
    \item Penalize shifting of collaboration days.
\end{enumerate}
A weighted multi-objective function (scalarized) is given by:
\[
\text{Maximize} \quad Z = \alpha \sum_{j \in J} \sum_{i \in I} \sum_{k \in K} x_{ijk} - \beta \sum_{i \in I} \sum_{k \in K} z_{ik} - \gamma \sum_{j \in J} (L_j - L_{\min}) - \delta \sum_{i \in I} \sum_{j \in J} u_{ij} - \eta \sum_{j \in J} \sum_{k \in K} w_{jk},
\]
where \(\alpha\), \(\beta\), \(\gamma\), \(\delta\) and \(\eta\) are positive weights reflecting the importance of each objective.
  
\textbf{Explanation:} The first term rewards high overall occupancy; the second penalizes team dispersion across floors; the third reduces daily occupancy variability; the fourth penalizes shifting (i.e., using \( u_{ij}=1 \)) to discourage unnecessary changes; and the fifth term penalizes the number of floors used in a day.

\section{Constraints}
\subsection*{(a) Floor Capacity Constraints}
For every floor $k$ and day $j$:
\[
\sum_{i \in I} x_{ijk} \leq F_k.
\]
\textbf{Explanation:} Ensures that no floor exceeds its available capacity.

\subsection*{(b) Employee Availability}
For every team $i$:
\[
 \sum_{k \in K} x_{ijk} \leq E_i.
\]
\textbf{Explanation:} A team cannot allocate more employees than it has, for any day \(j\).

\subsection*{(c) Fixed Seat Allocation Guarantee}
For every team $i$ and day $j$:
\[
\sum_{k \in K} x_{ijk} \geq S_{ij}.
\]
\textbf{Explanation:} Guarantees that each team receives its fixed seat allocation.

\subsection*{(d) Collaboration (Target) Requirements and Feasible Count Suggestion}
For every team $i$ and day $j$, we impose:
\[
\sum_{k \in K} x_{ijk} \leq T_{ij}.
\]
\[
\sum_{k \in K} x_{ijk} \geq 0.6\, T_{ij} \cdot (1 - u_{ij}).
\]
\textbf{Explanation:}  
\begin{itemize}[leftmargin=2em]
    \item The first inequality ensures that the allocation does not exceed the target collaboration requirement.
    \item The second inequality enforces that if the collaboration target is \emph{not shifted} (i.e., \(u_{ij}=0\)), the allocated count must be at least 60\% of the target. If \(u_{ij}=1\), the lower bound is relaxed (since $0.6\, T_{ij}\cdot0=0$), indicating that the requirement for that day is shifted.
\end{itemize}

\subsection*{(e) Fair Daily Load Constraint}
Define for every day $j$:
\[
L_j = \sum_{i \in I} \sum_{k \in K} x_{ijk}.
\]
Then enforce:
\[
L_{j} \geq L_{\min} \quad \forall j \in J,
\]
\[
L_j \leq \left(1 + \frac{P}{100}\right) \cdot L_{\min} \quad \forall j \in J.
\]
\textbf{Explanation:} These constraints maintain daily occupancy levels within a fixed range (e.g., within 20\% of the minimum day) to promote fairness across the week. \emph{Example:} If \(P=20\) and \(L_{\min}=1000\), then every \(L_j\) must be no more than 1200.

\subsection*{(f) Minimum Allocation if a Floor is Used}
For every team $i$, day $j$, and floor $k$, impose:
\[
x_{ijk} \geq 5 \cdot y_{ijk},
\]
\[
x_{ijk} \leq M \cdot y_{ijk}.
\]
\textbf{Explanation:} These ensure that if a team is assigned to a floor on a given day (i.e., \(y_{ijk}=1\)), then at least 5 employees must be allocated on that floor. If \(y_{ijk}=0\), then no employees are allocated there.

\subsection*{(g) Linking Daily Floor Use to Overall Floor Use}
For every team $i$, day $j$, and floor $k$, include:
\[
y_{ijk} \leq z_{ik}.
\]
\textbf{Explanation:} For every team $i$ and floor $k$, the above constraint ensures that if there exists any day $j$ such that \(y_{ijk}=1\), then:
\[
z_{ik} = 1.
\]
This linkage ensures that if a team uses a floor on any day, that floor is marked as used for the team over the planning horizon. Minimizing \(\sum_{i \in I} \sum_{k \in K} z_{ik}\) in the objective helps reduce multi-floor allocations.

\subsection*{(h) Linking Daily Floor Use to Number of Floors Used}
For every team \(i\), day \(j\) and floor \(k\), include:
\[y_{ijk} \leq w_{jk}\]
\textbf{Explanation:} For every day \(j\) and floor \(k\), the above constraint ensures that if there exists any \(i\) such that \(y_{ijk}=1\), then:
\[w_{jk}=1\] 
This linkage ensures that if a team uses a floor on a day, that floor is marked as used for that day in the planning horizon. Minimizing \(\sum_{j \in J}\sum_{k \in K} w_{jk}\) in the objective helps reduce the number of floors used on a day.

\subsection*{(i) Limiting Shifts Across Days}
For every team $i$, restrict the total number of shifted days:
\[
\sum_{j \in J} u_{ij} \leq 1.
\]
\textbf{Explanation:} This ensures that a team can have its collaboration shifted to a different day for at most one day over the planning horizon.

\section{Necessary Assumptions}
\begin{itemize}[leftmargin=2em]
    \item All parameters (employee counts, floor capacities, fixed seat requirements, and collaboration targets) are known in advance.
    \item Floor capacities are pre-adjusted to reflect social distancing norms (e.g., 25\%, 50\%, 70\% occupancy).
    \item Adjustments such as shifting collaboration days or floors are assumed to be implemented with the assistance of team leaders.
\end{itemize}

\section{Summary}
This extended linear formulation allocates fixed and floating seats to teams over days and floors while:
\begin{itemize}[leftmargin=2em]
    \item Ensuring no floor exceeds its capacity.
    \item Meeting each team’s fixed seat and collaboration requirements, with a mechanism to suggest a feasible count (at least 60\% of the target) when the target is too high.
    \item Allowing for the possibility to shift a team’s collaboration to a different day (for at most one day per team) if capacity is an issue.
    \item Penalizing the dispersion of a team across multiple floors.
    \item Promoting fairness in daily occupancy.
    \item Penalizing the number of floors used in a day.
\end{itemize}
The option to shift some team members to a different floor is inherently modeled by allowing allocation across floors while minimizing the number of floors used.

\end{document}
